\documentclass[type=doctor]{hustthesis} % 博士学位论文: doctor, 硕士学位论文: master

\hustsetup { % 设置之间一定不能有空行
  info =
  {
    title = { 中文标题 },
    title* = { English Title },
    degree = { academic }, % 学术型: academic, 专业型: professional
    degree* = { Engineering }, % 英文学科
    author = { 姓名 },
    author* = { Author },
    student-id = { D201234567 },
    clc = { }, % 中图分类号
    major = { 专业 },
    major* = { Major },
    supervisor = { 指导老师\quad{} 教授 },
    supervisor* = { Prof.~ Supervisor },
    date = { 2025-02-28 }, % 日期,不写这行的话会自动生成,写的话一定要按照这个格式
    committee = % 答辩委员会成员,不写的话会自动生成7行空白的答辩委员会成员
      { 张三 & 教授 & 华中科技大学 , 李四 & 教授 & 武汉大学 , 王五 & 教授 & 武汉理工大学 },
  },
  style =
  {
    toc-depth = section, % 目录章节显示深度: section 或 subsection
  },
}

\addbibresource{ref.bib} % % biblatex 参考文献文件

\begin{document}

\maketitle % 根据学位论文类型自动生成封面,就是下面这几个命令的组合
% \makezhtitle % 中文标题页
% \makecommittee % 答辩委员会成员 (硕士默认不生成,但可手动调用)
% \makeentitle % 英文标题页

\frontmatter

\begin{abstract}
  中文摘要
  \keywords 关键词1;关键词2;关键词3
\end{abstract}

\begin{abstract*}
  English Abstract
  \keywords* keyword1, keyword2, keyword3
\end{abstract*}

\tableofcontents

\mainmatter

\chapter{第一章}

\section{第一节}

文献标注示例 \TeX{}Book \cite{knuth1986TeXbook}.

\backmatter

\begin{acknowledgements} % 致谢
  感谢 hustthesis 文档类。
\end{acknowledgements}

\printbibliography

\appendix

\chapter{附录}

\end{document}
